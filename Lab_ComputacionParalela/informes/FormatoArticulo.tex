\documentclass[12pt]{article}
\usepackage{caption}

% Esto es para poder escribir acentos directamente:
\usepackage[utf8]{inputenc}
\usepackage[T1]{fontenc}

\usepackage{float}
\usepackage{enumerate} %-->

\usepackage[nottoc]{tocbibind}%-->

% Esto es para que el LaTeX sepa que el texto está en español españolao:
\usepackage[spanish]{babel}
\selectlanguage{spanish}%-->

\usepackage{titlesec}
\usepackage{titling}
\usepackage{xcolor}
\usepackage{hyperref}%compilar 2 veces
\usepackage{geometry}
\usepackage{listings}
\usepackage{pdfpages}
\usepackage{subfig}%figuras una alado de otraa


%% Paquetes de la AMS
\usepackage{amsmath, amsthm, amsfonts}%--->

%% Para añadir archivos con extensión pdf, jpg, png or tif
\usepackage{graphicx}%-->
\usepackage[colorinlistoftodos]{todonotes}%-->
%\usepackage[colorlinks=true, allcolors=blue]%{hyperref}%--->





\definecolor{commentsColor}{rgb}{0.13, 0.55, 0.13}
\definecolor{keywordsColor}{rgb}{0.000000, 0.000000, 0.635294}
\definecolor{stringColor}{rgb}{0.558215, 0.000000, 0.135316}
\definecolor{numerolineas}{rgb}{0.41,0.41,0.41}

\usepackage{verbatim}%coemntario begin%{} end

\usepackage{fancyhdr}%activar para usar encabezados esttilos funcy

%al final imprimir si solo utilizamos un printbibliografi  aqui el archivo bib
\usepackage[ backend=biber,style=apa]{biblatex}
\addbibresource{referencias.bib}%ojo con la ubicacion del bib y compilar con bibtex 

%\usepackage[top=1.5cm,bottom=1.0cm,left=1.25cm,right=1.25cm]%{geometry}%para todos las paginas




\providecommand{\keywords}[1]
{
  \small	
  \textbf{\textit{Keywords---}} #1
}

%% Primeroo escribimos el título
\title{Bucles Anidados }
\author{Angel Andres Bejar Merma\\
  \small Universidad Nacional de San Agustin\\
  \small abejar@unsa.com\\
  \small Ciudad de Arequipa
  %\date{\today} 
}

\begin{document}
%\newgeometry%{bottom=2.5cm,top=2.6cm,left=2.5cm,right=2.5cm}
%\restoregeometry
%\includepdf{9}

%
%paraa el report 


\maketitle

%% Aquí podemos añadir un resumen del trabajo (o del artículo en su caso) 
\begin{abstract}
    Estao es una plantilla simple para crear un articulo \LaTeX en español, con algunos comandos que se usarán frecuentemente para hacer tareas de la licenciatura en Física.
    \end{abstract}
    
    \keywords{one, two, three, four}
%\begin{abstract}



%\end{abstract}

%\vspace{5mm} %espacio vertical usar en imagen



\section{Parte 1}

Desarrollar un compilador descendente recursivo para evaluar expresiones booleanas,
dada la siguiente gramática.\ref{fig:1}




\begin{figure}[H]%estricto
\centering
%\includegraphics[width=15cm]{imagenes/.png}
\caption{}
\label{fig:1}
\end{figure}


\begin{figure}[H]%estricto
\centering
%\includegraphics[width=15cm]{imagenes/.png}
\caption*{}

\end{figure}


\begin{figure}[H]%estricto
\centering
%\includegraphics[width=15cm]{imagenes/.png}
\caption*{}

\end{figure}
\newpage
\section{Desarrollo}
\newpage



\section{Codigo}


%\end{samepage}


\begin{figure}[H]%estricto
\centering
%\includegraphics[width=15cm]{imagenes/.png}
\caption*{}

\end{figure}




\section{Anexo}



\begin{figure}[H]%estricto
\centering
%\includegraphics[width=15cm]{imagenes/.png}
\caption*{}

\end{figure}





\newpage
\section{Conclusion}

Este algoritmo conocido como el de construcción de
subconjuntos convierte un AFN en un AFD haciendo uso de



	


\vspace{20 mm}


 web1 \cite{naiouf2010procesamiento}




\printbibliography

\end{document}


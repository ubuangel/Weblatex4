\chapter{Introducción}
\hrule \bigskip \vspace*{1cm}
%------------------------------------------------------------------------

\section{Justificación} 
% ¿ Por qué vale la pena buscar lograr el objetivo planteado?  Se explica los detalles detrás de que la pregunta planteada aún no ha sido respondida, por ejemplo: citar que nadie presentó una solución satisfactoria o que existen soluciones contradictorias.


\section{Relevancia y/o Motivación de la Propuesta}

%¿cual es el ámbito en que esto se desenvuelve? y ¿que necesidad
%existe para motivar una investigación en tu tema?
%Se explica cómo la investigación contribuirá con algo valioso ya sea a nivel científico o social, el impacto que puede tener y por qué es importante que alguien se tome el tiempo para trabajar en esa problemática.


\section{Pregunta de Investigación}

%Es una pregunta que todavía no fue respondida satisfactoriamente. Debe ser clara, concisa, específica, neutral y enfocada. También deben ser lo suficientemente compleja como para que la pregunta requiera algo más que una respuesta de "sí" o "no", por lo que  usualmente comienza con: ¿Qué … ? ¿Cómo… ?






\section{Hipótesis}
%La hipótesis es una afirmación de la cual no se sabe si es verdadera o falsa. Es la posible respuesta a la pregunta de investigación. El trabajo de investigación consiste en probar la veracidad de la hipótesis. Es el corazón de la investigación, por lo que una buena hipótesis con evidencia de efectividad debe ser buscada.



\section{Objetivos}
% ¿Que pretendes obtener o resolver? y en los objetivos específicos
% detallar cada una de las actividades que realizaras, en este punto debes
% ser muy exacto y concreto(recuerda que en base a esto justificaras % si estas obteniendo resultados)

\subsection{Objetivo General}
%El objetivo debe ser directamente verificable al final del trabajo de investigación, un buen objetivo demostrará que la hipótesis que está siendo probada es verdadera o no. Tenga en cuenta que el objetivo no debe ser confundido con el tema de investigación. El objetivo general y los objetivos específicos deben ser formulados de manera que se puedan verificar al final del trabajo, comienzan con verbos en infinitivo, por ejemplo: demostrar, mejorar, analizar.

\subsection{Objetivos específicos}
%Los objetivos específicos no son etapas de la investigación, son subproductos que serán el resultado del proceso de verificación del objetivo general. Los objetivos específicos deben indicar una contribución.




%\section{Organización de la tesis}

% Una breve descripción de cada uno de los capítulos que estas
% desarrollando desde el CAP 2 hasta el capitulo antes del apéndice.
